% \documentclass[slidestop,compress]{beamer}
% \setlength{\TPHorizModule}{10in}
% \usepackage[orientation=landscape,size=custom, width=21, height=18]{beamerposter}
% 	\usetheme{simple}
\documentclass[24pt]{minimal}
\usepackage[paperwidth=21in, paperheight=18in, margin=0.25in]{geometry}
\usepackage{lmodern}
\usepackage{lipsum}
\usepackage{calc}
\usepackage{graphicx}
\usepackage{color}
\setlength{\parindent}{0pt}
% \usepackage[scale=2]{ccicons}

\begin{document}
%\begin{textblock}{10in}{0,0}
%\begin{block}{\centering Introduction}
%\end{block}
%\end{textblock}
\parbox{10in}{
The amateur and university rocketry communities are rapidly reaching higher altitudes with more sophisticated rockets. However, most groups are still using heavy airframes made of metal or fiberglass. Commercial off-the-shelf airframes are either too expensive for low-budget university groups or too small to use as a platform for high altitude experiments. 
A capstone team of mechanical engineering seniors at Portland State University has developed a low-weight, modular carbon fiber airframe as an open-hardware technology for university rocketry. 
This project continues the work of a 2014 capstone team, who developed a carbon fiber layup process with promising results. 
This will enable low-budget groups like the Portland State Aerospace Society to explore high altitude science and compete in the university space race.  


% \begin{figure}[h]
\def\svgwidth{\linewidth}
\input{moduleDiagram.pdf_tex}
% \caption{
% 	Diagram of the male end of a module. 
% 	The CF (1) is bonded to the honeycomb core (3) and the aluminum coupling ring (4) using structural adhesive (2). 
% 	The adhesive also serves as a protective coating for the CF and provides a smooth outer surface. 
% 	See figure \ref{fig:coupon} for a picture of this design.
% 	}
\label{fig:moduleDiagram}
\hfill
\input{coupon.pdf_tex}
% \caption{
% 	A cut-away sample displaying the layers used in the LV3 sandwich shells. 
% 	See figure \ref{fig:moduleDiagram} for a diagrammatic depiction.
% 	}
\label{fig:coupon}
% \end{figure}
}

\end{document}
