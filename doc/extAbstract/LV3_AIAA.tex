\documentclass{aiaa-tc}% insert '[draft]' option to show overfull boxes
\title{Design and Manufacture of an Open-Hardware 
 	University Rocket Airframe using Carbon Fiber}

\author{
Joseph Shields, Brandon Bonner, Leslie Elwood, Erik Nelson, and Jacob East
	\thanks{Portland State University, Portland, OR 97201}
 }
% Data used by 'handcarry' option if invoked
\AIAApapernumber{2016}
\AIAAconference{Conference Name, Date, and Location}
\AIAAcopyright{\AIAAcopyrightD{2016}}
 
% Define commands to assure consistent treatment throughout document
\newcommand{\eqnref}[1]{(\ref{#1})}
\newcommand{\class}[1]{\texttt{#1}}
\newcommand{\package}[1]{\texttt{#1}}
\newcommand{\file}[1]{\texttt{#1}}
\newcommand{\BibTeX}{\textsc{Bib}\TeX}
\newcommand{\cots}{commercial off-the-shelf}
\begin{document}
\maketitle

\begin{abstract}
The amateur and university rocketry communities are rapidly reaching higher altitudes with more sophisticated rockets. However, most groups are still using heavy airframes made of metal or fiberglass. Commercial off-the-shelf airframes are either too expensive for low-budget university groups or too small to use as a platform for high altitude experiments. A capstone team of mechanical engineering seniors at Portland State University is developing a low-weight, modular carbon fiber airframe as an open-hardware technology for university rocketry. This will enable low-budget groups like the Portland State Aerospace Society to explore high altitude science and compete in the university space race. 
\end{abstract}
%\section*{Nomenclature}

\section{Introduction}
The Portland State Aerospace Society is an interdisciplinary group of engineering students and alumni of Portland State University with the long term goal of putting a cubesat into orbit with their own rocket. 
% info about cubesat market growth: http://www.spaceref.com/news/viewpr.html?pid=44940 
Their current airframe, named LV2, has served for over 12 years and hosted experiments ranging from custom patch antennas and long range WiFi technology to GPS navigation and a cold gas reaction controll system. 

\section{Significance}


\section{Plan of Work}

\end{document}
